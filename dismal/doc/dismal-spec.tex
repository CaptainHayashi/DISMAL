\documentclass[a4paper]{scrbook}
\usepackage{geometry}

\usepackage{fancyhdr}

\usepackage[T1]{fontenc}

\usepackage{avant}
\usepackage{charter}

\usepackage{lastpage}
\usepackage{wrapfig}

\usepackage{ulem}

\usepackage{longtable}

%\usepackage{amsmath, amsthm, amssymb}

\ifx\pdfoutput\undefined
% we are running LaTeX, not pdflatex
\usepackage{graphicx}
\usepackage[colorlinks=true]{hyperref}
\else
% we are running pdflatex, so convert .eps files to .pdf
\usepackage[pdftex]{graphicx}
\usepackage[pdftex,colorlinks=true]{hyperref}
\usepackage{epstopdf}
\fi 

\newcommand{\tablehead}[1]{\textbf{#1}}
\newcommand{\code}[1]{\texttt{#1}}

\normalem

\begin{document}

\frontmatter

\begin{titlepage}

  \begin{center} 
    \huge{DISMAL}\\
    \bigskip
    \large{Drawing/Input/Sound Meta-Abstraction Layer}
    \vfill
    Technical Specification and User's Guide
  \end{center}

\end{titlepage}

%\fancyhf{}

\pagestyle{fancyplain}

\tableofcontents

\mainmatter

\pagestyle{fancy}

\part{Specification}

\chapter{Goals}

DISMAL (an acronym for Drawing/Input/Sound Meta-Abstraction Layer) is
an attempt to make a small, highly portable game engine framework for
2D games programmed in C.

DISMAL is termed a \emph{meta-abstraction layer} as it often relies on
the work of other abstraction layers, such as SDL, to interface with
the target hardware.

The layer will provide a more high-level outlook than most of its
backends, hiding the complexity of creating a screen, handling input,
managing resources and providing timing features.

\section{Specific Targets}

\subsection{Generic}

\begin{itemize}
  \item The layer must be written in compliant C89, with the possible
    exception of code expected to be compiled by one compiler (for
    example, target driver code).
  \item The layer must run on Windows 95 and above.
  \item For the first milestone, the layer must be sufficient to run a
    SameGame clone on target code for SDL, protected mode DOS and
    AmigaOS 3.1.
  \item The entire code base must be well-documented through use of
    both regular and Doxygen special comments.
\end{itemize}

\subsection{Modules}

\begin{itemize}
  \item The layer code will be organised into modules. For the first
    milestone, these will be selected at compile-time by use of
    conditional compilation.
  \item A \emph{base} module will be produced for each target
    family. Each other module will depend on a particular base, to
    allow large libraries to share init/deinit code and prevent
    unworkable module combinations.
  \item For the first milestone, three base modules will be created:
    SDL (Simple DirectMedia Layer 1.2, multi-platform), Amiga68k
    (AmigaOS 3.x and above on 68020 and above processors) and DOS
    (32-bit extended DOS on 80386 and above).
  \item The following other module types will eventually be produced:
    \begin{itemize}
      \item GFX (low-level graphics interface, including image
        resource handler)
      \item SFX (sound and music interface)
      \item Input (keyboard and mouse interface)
      \item State (finite state machine engine)
      \item GUI (a more high-level user interface layer)
      \item Time (timing and FPS, if available)
      \item Config (argument vector, config file and Windows registry
        reader)
    \end{itemize}
  \item In addition to the modules, a small amount of DISMAL's
    platform-independent functionality will be kept outside the module
    system and compiled into all programs. This includes DISMAL's main
    init/deinit code.
\end{itemize}

\subsection{GFX}

\begin{itemize}
  \item The layer must be able to report to its parent code if the GFX
    target is a low-resolution target. Low-resolution is defined as
    320x200 8-bit.
  \item The layer must be able to adapt graphics intended for 320x200
    8-bit to higher resolutions if the parent code cannot. This
    includes double-scaling graphics and co-ordinates and
    letterboxing.
  \item All targets of the layer must support the loading of
    256-colour PCX files, with an optional transparent palette index
    (either given explicitly or stated as a RGB triple to match, for
    example magenta 255, 0, 255), as a fallback when specific target
    image files are not present.
  \item The layer must always provide a logical palette of 256
    colours, which is attached to the physical palette in
    low-resolution modes.
  \item The layer will work in terms of 8bpp colour components,
    quantising these when necessary (for example to 6bpp VGA).
  \item The PCX handler should be able to crudely remap colours to the
    existing logical palette by comparing average colour values.
\end{itemize}

\subsection{Sound}

\begin{itemize}
  \item All targets should be capable of playing back ProTracker
    Module (.mod) format files as music.
\end{itemize}

\subsection{Input}

\begin{itemize}
  \item The input layer will make use of callbacks to provide input
    events to other parts of the code. These callbacks will be
    provided along with input filters, so that only relevant
    information triggers them (for example, mouse button code can
    request that only mouse events trigger a callback).
  \item ASCII key events and non-ASCII (``special'') key events will
    be segregated. The layer will pass all ASCII key events
    unmodified, but special keys will be passed instead as constants
    referring to their meaning (eg Escape key, arrow keys, Enter
    key). These will trigger separate callbacks.
\end{itemize}

\subsection{State}

\begin{itemize}
  \item DISMAL will provide a generic, well-commented main loop,
    pulling in all enabled modules using \#ifdef, which can be used
    directly in code or copied and pasted into the parent program's
    logic and modified.
  \item DISMAL will provide the framework for a finite state machine
    model of application programming, using function pointers to
    achieve polymorphism without the use of C++.
\end{itemize}

\part{Technical}

\part{User's Guide}

\chapter{Preparation}

DISMAL is, at the time of writing, maintained and distributed as a
directory of source code with the intention of being included in a
project and compiled as part of the project.

Therefore, in order to use DISMAL, place it in the directory tree of
the DISMAL-using project, and chain the DISMAL makefile into the
project makefile, or otherwise incorporate the necessary DISMAL
sources into the project compiling process.

\chapter{Initialisation and deinitialisation}

\subsection{Initialising DISMAL}

The easiest way to use DISMAL is with code similar to the following:

\begin{figure}[h]
  \begin{verbatim}
if (dm_init() == DM_SUCCESS) {
  ...code goes here...
  dm_cleanup();
} else {
  fprintf(stderr, "Could not initiate DISMAL!\n");
}
  \end{verbatim}

\caption{DISMAL initialisation code.}

\end{figure}

What is, or is not, initialised depends on how DISMAL was compiled. By
default, all compiled-in modules will be initialised.

In order to gain finer control over the loading of modules, one can
use the function \verb:dm_set_module_list: to change the module
list. An example is given below.

\begin{figure}[h]
  \begin{verbatim}

dm_set_module_list(DM_MOD_GFX | DM_MOD_SFX); /* Init GFX and sound only. */

if (dm_init() == DM_SUCCESS) {
  /* Code goes here */
  dm_cleanup();
} else {
  fprintf(stderr, "Could not initiate DISMAL!\n");
}
  \end{verbatim}

\caption{DISMAL initialisation code with a custom configuration.}

\end{figure}

\textbf{Note:} If \verb:init_modules: includes a module but not its
prerequisite modules (for example, UI but not GFX), those modules will
automatically be loaded.

\textbf{Note:} DISMAL returns \verb:DM_SUCCESS: if there were no fatal
errors reached while loading. A \verb:DM_SUCCESS: during init does not
necessarily mean everything has loaded - see the next secion.

\subsection{Getting information about DISMAL}

One of the first things you may want to do after initialising DISMAL
is to check a few things about the engine state. DISMAL exports its
current status through several functions, documented here.

DISMAL regards errors loading the SFX module as non-fatal - that is,
\verb:dm_init: will report \verb:DM_SUCCESS: even if the SFX loader
has failed. This is because sound is generally an optional extra in
games (which is DISMAL's intended purpose), whereas most other modules
are critical to game functions. However, all other module failures
will cause DISMAL to fail to load.

Therefore, you may wish to check to see if the sound module has loaded
and, if not, why not. This can be done by checking the output of the
function \verb:dm_get_module_list:. This will return an integer with
the lags for each initialised module set.

\begin{figure}[h]
  \begin{verbatim}

int modlist;

modlist = dm_get_module_list(); 

if (modlist & DM_MOD_SFX) {
  /* Sound code goes here */
} else {
  /* No sound */
}
  \end{verbatim}

\caption{Checking to see if the sound module loaded.}

\end{figure}

\chapter{Using the graphics subsystem}

\section{High-res and low-res}

Due to DISMAL targetting widely disparate 

\section{Loading images}

DISMAL provides two methods for loading an image: direct image
loading, and using an image resource manager.

\subsection{Direct loading}

Direct loading has the advantage of requiring less work to use, but
places the burden of ensuring that loaded images are appropriate for
all graphics environments (high-res/low-res, for example) on the
calling code.

To load a non-transparent image, use the function
\verb:dm_load_image:. This takes exactly one parameter: the name of
the image. The name of the image is later used to reference the image
(for example, 

\end{document}
